\thispagestyle{empty}
\setcounter{page}{3}
\section*{\centering РЕФЕРАТ}

\vspace{0.6cm}
Пояснительная записка к курсовой работе содержит: \totalpages{} страниц,
\totalfigures{} рисунков, 5 ссылок и \totalsections{} приложения.

Цель исследования -- разработать законченный проект, представляющий собой
клавиатурный тренажер для слепого десятипальцевого метода набора.

Особенность разрабатываемого приложения состоит в применении кроссплатформенных
технологий, что позволит запускать приложение на всех популярных платформах.


В данном курсовом проекте в качестве технологии построения кроссплатформенных
приложений используется Qt Framework в виде привязки к языку Python(PySide),
a также базы данных SQLite.
\newline
\newline
PYTHON, PYSIDE, QT, SQLITE, GIT, ВИДЖЕТ, ТРЕНАЖЕР 
