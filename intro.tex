\section*{\centering ВВЕДЕНИЕ}
\addcontentsline{toc}{section}{ВВЕДЕНИЕ}

\vspace{0.6cm}

В настоящее время индустрия разработки программного обеспечения переживает
бурный рост. В связи с низким порогом вхождения и количественным увеличением
числа разработчиков всё чаще возникают вопросы о качестве создаваемого программного
обеспечения.

Современный разработчик должен обладать знаниями предметной области, для которой
разрабатывается приложение, а также используемых инструментов: языка программирования,
среды разработки, отладчика, системы контроля версий и~т.~д.

На этапе принятия решения о начале разработки приложения немаловажным является
выбор используемых технологий. Правильно выбранный набор современных технологий
позволяет создавать кроссплатформенные приложения в разумные сроки.

При разработке архитектуры приложения огромное значение имеет знание и применение
разработчиком шаблонов проектирования --- известных решений проблем, часто
возникающих при разработке различных приложений.

Однако даже соблюдение рассмотренных положений само по себе не приведёт
к созданию хороших приложений. Главным остаётся наличие у разработчика
необходимого опыта использования своих знаний. Очевидное решение
данной проблемы --- участие разработчика в реальных проектах, ровно
как и создание собственных, пусть и лишь в целях обучения, а не применения
конечными пользователями.